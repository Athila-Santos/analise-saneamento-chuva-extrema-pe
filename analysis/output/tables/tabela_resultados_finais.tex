%%%%%%%%%%%%%%%%%%%%%%%%%%%%%%%%%%%%%%%%%%%%%%%%%%%%%%%%%%%%%%%%%%%%%%%%%%%%%%%%
%                                                                              %
%               ARQUIVO LaTeX - MODELO COMPLETO PARA TABELA EM PAISAGEM          %
%                                                                              %
%%%%%%%%%%%%%%%%%%%%%%%%%%%%%%%%%%%%%%%%%%%%%%%%%%%%%%%%%%%%%%%%%%%%%%%%%%%%%%%%

% --- 1. CLASSE DO DOCUMENTO E PACOTES ESSENCIAIS ---
% Define o tipo de documento (artigo), tamanho da fonte padrão (12pt) e papel (a4)
\documentclass[12pt, a4paper]{article}

% --- Pacotes para codificação e idioma ---
% Permite o uso de caracteres acentuados diretamente no código (ç, ã, é)
\usepackage[utf8]{inputenc}
\usepackage[T1]{fontenc}
% Define o idioma para português do Brasil (traduções automáticas e hifenização)
\usepackage[brazil]{babel}

% --- Pacotes para a formatação da tabela ---
% Pacote para criar páginas em modo paisagem (landscape)
\usepackage{pdflscape}
% Pacote para incluir e manipular imagens e caixas (usado para \resizebox)
\usepackage{graphicx}
% Pacote para criar linhas de tabelas com visual profissional
\usepackage{booktabs}

% --- Pacotes para formatação geral (recomendado) ---
% Pacote para ajustar as margens da página
\usepackage{geometry}
\geometry{a4paper, margin=2.5cm} % Define margens de 2.5cm em todos os lados


% --- Informações do Documento (Opcional) ---
\title{Exemplo de Tabela em Página de Paisagem}
\author{Seu Nome Aqui}
\date{\today}


% --- 2. INÍCIO DO DOCUMENTO ---
\begin{document}

% Cria a página de título com as informações acima
\maketitle

\section{Introdução}
Este é um exemplo de um documento LaTeX. O texto normal, como este parágrafo, é exibido em modo retrato (vertical).

A seguir, a tabela de resultados será apresentada. Como ela é muito larga, foi colocada em uma página separada e em modo paisagem (horizontal) para garantir que todo o conteúdo seja visível e bem formatado.


%%%%%%%%%%%%%%%%%%%%%%%%%%%%%%%%%%%%%%%%%%%%%%%%%%%%%%%%%%%%%%%%%%%%%%%%
%                  AQUI COMEÇA A TABELA EM PAISAGEM                    %
%%%%%%%%%%%%%%%%%%%%%%%%%%%%%%%%%%%%%%%%%%%%%%%%%%%%%%%%%%%%%%%%%%%%%%%%

\begin{landscape} % Inicia o ambiente que coloca tudo dentro dele em modo paisagem

% Usar [p] é uma boa prática para tabelas de página inteira.
% O LaTeX tentará colocar em uma página separada.
\begin{table}[p]
\centering
\caption{Resultados da Análise de Regressão} % Adicionei uma legenda à tabela

% O comando \resizebox ajusta o tamanho da tabela para a largura do texto.
% O '!' mantém a proporção original de altura e largura.
\resizebox{\linewidth}{!}{%
\begin{tabular}{@{\extracolsep{5pt}}lccccccccc}
\toprule % Linha superior profissional (pacote booktabs)
 & \multicolumn{1}{c}{(1) Referência} & \multicolumn{1}{c}{(2) Choque (Intens.)} & \multicolumn{1}{c}{(3) Saneam. (Água)} & \multicolumn{1}{c}{(4) Saneam. (Elo Fraco)} & \multicolumn{1}{c}{(5) Desfecho (Arbov.)} & \multicolumn{1}{c}{(6) Desfecho (PIB)} & \multicolumn{1}{c}{(7) Dinâmica Temp.} & \multicolumn{1}{c}{(8) Desfecho (Esquisto.)} & \multicolumn{1}{c}{(9) Desfecho (Febre Tif.)}  \\
\cmidrule(lr){2-2} \cmidrule(lr){3-3} \cmidrule(lr){4-4} \cmidrule(lr){5-5} \cmidrule(lr){6-6} \cmidrule(lr){7-7} \cmidrule(lr){8-8} \cmidrule(lr){9-9} \cmidrule(lr){10-10} % Linhas parciais para os cabeçalhos
 & (1) & (2) & (3) & (4) & (5) & (6) & (7) & (8) & (9) \\
\midrule % Linha intermediária (pacote booktabs)
 Cob. Água & & & 0.001$^{}$ & & -0.006$^{}$ & & & & 0.001$^{}$ \\
& & & (0.002) & & (0.007) & & & & (0.001) \\
 Cob. Esgoto & 0.002$^{}$ & 0.002$^{}$ & & & & 0.001$^{**}$ & 0.001$^{}$ & -0.004$^{}$ & \\
& (0.002) & (0.002) & & & & (0.001) & (0.002) & (0.005) & \\
 Constante & 0.991$^{*}$ & 1.007$^{*}$ & 0.952$^{*}$ & 0.964$^{*}$ & 3.578$^{**}$ & 9.297$^{***}$ & 1.226$^{**}$ & -6.188$^{***}$ & 0.418$^{*}$ \\
& (0.540) & (0.540) & (0.541) & (0.540) & (1.686) & (0.016) & (0.564) & (1.497) & (0.245) \\
 Densidade Pop. & -0.001$^{***}$ & -0.001$^{***}$ & -0.001$^{***}$ & -0.001$^{***}$ & 0.001$^{}$ & -0.000$^{}$ & -0.001$^{***}$ & -0.000$^{}$ & 0.000$^{}$ \\
& (0.000) & (0.000) & (0.000) & (0.000) & (0.001) & (0.000) & (0.000) & (0.001) & (0.000) \\
 Dias Chuva Extrema & 0.000$^{}$ & & 0.000$^{}$ & 0.000$^{}$ & -0.003$^{**}$ & -0.000$^{}$ & 0.000$^{}$ & -0.002$^{}$ & -0.000$^{}$ \\
& (0.000) & & (0.000) & (0.000) & (0.001) & (0.000) & (0.000) & (0.001) & (0.000) \\
 Chuva * Esgoto & 0.000$^{*}$ & & & & & & & & \\
& (0.000) & & & & & & & & \\
 Volume * Esgoto & & 0.000$^{}$ & & & & & & & \\
& & (0.000) & & & & & & & \\
 Chuva * Água & & & 0.000$^{***}$ & & & & & & \\
& & & (0.000) & & & & & & \\
 Chuva * Elo Fraco & & & & 0.000$^{**}$ & & & & & \\
& & & & (0.000) & & & & & \\
 Chuva * Água (Arbov.) & & & & & 0.000$^{**}$ & & & & \\
& & & & & (0.000) & & & & \\
 Chuva * Esgoto (PIB) & & & & & & -0.000$^{}$ & & & \\
& & & & & & (0.000) & & & \\
 Chuva * Esgoto (Lag) & & & & & & & 0.000$^{*}$ & & \\
& & & & & & & (0.000) & & \\
 Chuva * Esgoto (Esquisto.) & & & & & & & & -0.000$^{}$ & \\
& & & & & & & & (0.000) & \\
 Chuva * Água (Febre Tif.) & & & & & & & & & 0.000$^{}$ \\
& & & & & & & & & (0.000) \\
 ln(PIB pc) & -0.058$^{}$ & -0.059$^{}$ & -0.055$^{}$ & -0.054$^{}$ & 0.052$^{}$ & & -0.086$^{}$ & 0.789$^{***}$ & -0.043$^{}$ \\
& (0.058) & (0.058) & (0.058) & (0.058) & (0.180) & & (0.060) & (0.160) & (0.026) \\
 ln(Taxa Lepto) [t-1] & & & & & & & -0.013$^{}$ & & \\
& & & & & & & (0.017) & & \\
 Saneam. (Elo Fraco) & & & & 0.001$^{}$ & & & & & \\
& & & & (0.002) & & & & & \\
 Volume Chuva Extrema & & 0.000$^{}$ & & & & & & & \\
& & (0.000) & & & & & & & \\
\midrule % Linha intermediária (pacote booktabs)
 Observations & 3700 & 3700 & 3700 & 3700 & 3700 & 3700 & 3515 & 3700 & 3700 \\
 N. of groups & 185 & 185 & 185 & 185 & 185 & 185 & 185 & 185 & 185 \\
 $R^2$ & 0.007 & 0.007 & 0.010 & 0.007 & 0.003 & 0.003 & 0.006 & 0.008 & 0.002 \\
 Residual Std. Error & 0.041 (df=3491) & 0.040 (df=3491) & 0.047 (df=3491) & 0.041 (df=3491) & 0.088 (df=3491) & 0.007 (df=3492) & 0.036 (df=3306) & 0.119 (df=3491) & 0.009 (df=3491) \\
\bottomrule % Linha inferior profissional (pacote booktabs)
\multicolumn{10}{l}{\textit{Nota: Efeitos fixos de município e ano incluídos em todos os modelos.}} \\
\multicolumn{10}{l}{\textit{Erros padrão entre parênteses. $^{*}$p$<$0.1; $^{**}$p$<$0.05; $^{***}$p$<$0.01}} \\
\end{tabular}
} % Fecha o \resizebox
\end{table}

\end{landscape} % Termina o ambiente de paisagem

%%%%%%%%%%%%%%%%%%%%%%%%%%%%%%%%%%%%%%%%%%%%%%%%%%%%%%%%%%%%%%%%%%%%%%%%
%                     FIM DA TABELA EM PAISAGEM                        %
%%%%%%%%%%%%%%%%%%%%%%%%%%%%%%%%%%%%%%%%%%%%%%%%%%%%%%%%%%%%%%%%%%%%%%%%


\section{Conclusão}
Após a apresentação da tabela, o documento retorna automaticamente para a orientação de retrato padrão para as seções e textos subsequentes, como esta conclusão.


% --- 3. FIM DO DOCUMENTO ---
\end{document}